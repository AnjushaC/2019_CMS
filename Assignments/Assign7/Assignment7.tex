\documentclass{article}
\usepackage{sivaSAFRANshort}
\chead{Computer Modelling and Simulation: Assignment $7$}
\begin{document}
	\begin{enumerate}
		\item
		In class, in the proof of the Euler-Maclaurin formula, we left the following as an exercise.
		$$\dint_a^b \dfrac{B_{2m+2}(\{t\})}{(2m+2)!} g^{(2)}(t)dt = \dint_a^b \dfrac{B_{2m}(\{t\})}{(2m)!} g(t)dt + \dfrac{b_{2m+2}}{(2m+2)!} \bkt{g'(b)-g'(a)}$$
		Use integration by parts to prove the above. Note that $B_k(x)$ are the Bernoulli polynomials and $\{t\}$ is the fractional part of $t$.
		\item
		In class, we obtained that
		$$n! \sim C \sqrt{n} \bkt{\dfrac{n}{e}}^n, \,\,\,\,\,\, \bkt{\spadesuit}$$
		The notation $f(n) \sim g(n)$ means that $\lim_{n \to \infty} \dfrac{f(n)}{g(n)} = 1$. In this exercise, we will obtain $C$ by following the sequence of steps.
		\begin{itemize}
			\item
			Show that
			$$I_n=\dint_0^{\pi/2} \sin^{n}(x)dx = \begin{cases}
			\dfrac{\pi}{2^{2n+1}} \dbinom{2n}n & \text{ if $n$ is even}\\
			\dfrac{2^{2n}}{2n+1} \dfrac1{\dbinom{2n}n} & \text{ if $n$ is odd}\\
			\end{cases}$$
			\item
			Show that
			$$\lim_{n \to \infty}\dfrac{I_{2n-1}}{I_{2n+1}} = 1$$
			\item
			Prove that $I_n$ is a monotone decreasing sequence, i.e.,
			$$I_{2n+1} < I_{2n} < I_{2n-1}$$
			\item
			Conclude that
			$$\lim_{n \to \infty}\dfrac{I_{2n}}{I_{2n+1}} = 1$$
			\item
			Hence, conclude that
			$$\dbinom{2n}n \sim \dfrac{4^n}{\sqrt{n \pi}}$$
			\item
			Expand the central binomial coefficient and derive that $C = \sqrt{2\pi}$ using $\bkt{\spadesuit}$.
		\end{itemize}
		\item
		Let $I = \dint_0^1 \exp\bkt{x^2}dx$. The exact value of the integral accurate upto $16$ digits is $1.46265174590718161$. Compute the integral using the following methods by sub-dividing $[0,1]$ into $N \in \{2,4,8,16,32,64,128,256,512,1024\}$ panels. Plot the decay of the absolute error as a function of $N$ on a log-log plot.
		\begin{itemize}
			\item
			Midpoint rule
			\item
			Trapezoidal rule
			\item
			Trapezoidal rule with end point correction using the first derivative
			\item
			Trapezoidal rule with end point correction using the first and second derivative
		\end{itemize}
		Repeat the same using Gauss Legendre quadrature with $N \in \{2,3,4,\ldots,51\}$ and plot the absolute error as a function of $N$ on a log-log plot. For an accuracy of $10^{-12}$, report the number of nodes required by
		\begin{itemize}
			\item
			Midpoint rule
			\item
			Trapezoidal rule
			\item
			Trapezoidal rule with end point correction using the first derivative
			\item
			Trapezoidal rule with end point correction using the first and second derivative
			\item
			Gauss Legendre quadrature
		\end{itemize}
		\item
		Evaluate
		$$I = \dint_0^2 \dfrac{e^{-x}}{\sqrt{x}}dx$$
		using the following methods
		\begin{itemize}
			\item
			Use the midpoint rule to avoid the singularity of the integrand at $x=0$
			\item
			Make a change of variable $x=t^2$ and again use the midpoint rule to evaluate the resulting integral
		\end{itemize}
		Compare the accuracy of the two methods by plotting the error as a function of $N$ (number of panels) on a lo-log scale, where $N \in \{2,4,8,16,32,64,128,256,512,1024\}$.
	\end{enumerate}
\end{document}